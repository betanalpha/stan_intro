\chapter{Probability in Theory}

While logic provides a framework for how we can describe systems 
with certainty, probability theory allow us to assign \emph{uncertainty} 
to logical propositions. Jaynes noted that "[Logic] is the kind of
reasoning we would like to use all the time; but... in almost all 
the situations confronting us we do not have the right kind of 
information to allow this kind of reasoning." (Jaynes Ch. 1) 
When we are uncertain about our target system then we cannot
guarantee that any particular logical statement is absolutely
true or false. 

We begin this chapter with an introduction of measures, which allow us
to assign arbitrary values to propositions, and then probability measures, 
which limit those value to probabilities that quantify uncertainty. 
We end with a discussion of conditional probability distributions 
over implications.

Unfortunately, the abstraction of probability theory means that we 
cannot rely on explicit examples, and consequently visualizations, 
to support the definitions and manipulations introduced in this chapter.  
That context will have to wait until the next chapter where we discuss 
how to implement probability theory in practice.

\section{Measures}

The first step towards assigning uncertainty to logical statements is to
assign \emph{any} positive value.  A \emph{measure} accomplishes this
by assigning a positive value to all events in the chosen event space,
%
\begin{equation*}
\mu : \mathcal{E} \! \left( \Theta \right) \rightarrow \mathbb{R}^{+}.
\end{equation*}
%
The term measure often serves multiple uses depending on context.
It can, for example, refer to the object which assigns values to each
event or the actual value assigned to a particular event, as in ``the 
measure of the event $E$ is $\mu \! \left( E \right)$''.

Because a measure assigns values to only events, we need both a
sample space and an event space in order to define measure.  In
other words, we can define measures only on measurable spaces.  
This is the ultimate motivation for this term, as measurable spaces 
are most often introduced not as representations of logical models
but only as a precursor to measures themselves.

Provided that the measure of each event is finite, 
%
\begin{equation*}
\mu \! \left[ E \right] < \infty, \, \forall E \in \EV{\Theta},
\end{equation*}
%
the assignment of measures is naturally compatible with Boolean
operations.  For example, the measure of conjunctions and
disjunction are related as
%
\begin{equation*}
\mu \! \left( E_{1} \cup E_{2} \right)
= 
\mu \! \left( E_{1} \right) + \mu \! \left( E_{2} \right)
- \mu \! \left( E_{1} \cap E_{2} \right),
\end{equation*}
%
and negations satisfy,
%
\begin{equation*}
\mu \! \left( E \right)
= 
\mu \! \left( \Theta \right) - \mu \! \left( E^{c} \right)
=
1 - \mu \! \left( E^{c} \right).
\end{equation*}
%
By definition all events have finite measure, and all of these results
hold, provided that the trivial event has finite measure, 
$\mu \! \left( \Theta \right) < \infty$.

Measures can also be used to integrate functions over the sample
space in a procedure known as \emph{Lebesgue integration},
%
\begin{equation*}
\mathcal{I}_{\mu} : 
\mathcal{F} \! \left( \Theta \right) \rightarrow \mathbb{R},
\end{equation*}
%
where $\mathcal{F} \! \left( \Theta \right)$ is the collection of 
well-behaved functions $f : \Theta \rightarrow \mathbb{R}$.  In 
fact, we can also consider measure assignments themselves 
as Lebesgue integrals,
%
\begin{equation*}
\mu \! \left( E \right)
= 
\mathcal{I}_{\mu} \! \left[ \mathbb{I}_{E} \right],
\end{equation*}
%
where the \emph{indicator function} of the event $E$, $\mathbb{I}_{E}$,
is defined as
%
\begin{equation*}
\mathbb{I}_{E} \! \left( \theta \right)
= 
\left\{
\begin{array}{rr}
0, & \theta \notin E \\
1, & \theta \in E
\end{array}
\right. .
\end{equation*}
%
In particular, the Lebesgue integral of every well-behaved function
completely specifies a given measure.

In some cases Lebesgue integration can also be used to emulate 
one measure with other.  Consider two measures $\mu_{1}$
and $\mu_{2}$ where $\mu_{2} \! \left( E \right) = 0$ whenever 
$\mu_{1} = 0$.  We say that any measure satisfying this property 
is \emph{absolutely continuous} with respect to $\mu_{1}$.  

When $\mu_{2}$ is absolutely continuous with respect to
$\mu_{1}$, the Lebesgue integral of any well-behaved
function, $f$, can be recovered as a modified Lebesgue 
integral with respect to $\mu_{1}$,
%
\begin{equation} \label{eqn:rnd_def}
\mathcal{I}_{\mu_{2}} \! \left[ f \right]
=
\mathcal{I}_{\mu_{1}} \! \left[ \nu \cdot f \right],
\end{equation}
%
where $\nu \in \mathcal{F} \! \left( \Theta \right)$ is a bounded 
function denoted the \emph{Radon-Nikodym derivative} of 
$\mu_{2}$ with respect to $\mu_{1}$.  

The Radon-Nikodym derivative is often written as
%
\begin{equation*}
\nu = \frac{ \mathrm{d} \mu_{2} }{ \mathrm{d} \mu_{1} }
\end{equation*}
%
to make the relevant measures more explicit.  Moreover, it typical to
abuse notation a bit and write
%
\begin{equation*}
\mu_{2} = 
\frac{ \mathrm{d} \mu_{2} }{ \mathrm{d} \mu_{1} } 
\mu_{1}
\end{equation*}
%
as shorthand for the defining relationship \eqref{eqn:rnd_def}.

\section{Pushforward Measures}

Like events, measures can be defined with respect to many 
different sample spaces.  If $s : \Theta \rightarrow \Omega$ 
is a measurable map and $\mu_{\Theta}$ is a probability 
distribution defined over $\left( \Theta, \EV{\Theta} \right)$, 
then we can define an equivalent \emph{pushforward} measure 
over $\left( \Omega, \EV{\Omega} \right)$ by assigning values as
%
\begin{equation*}
s_{*} \mu_{\Theta} \! \left( E_{\Omega} \right)
\equiv
\mu_{\Theta} \! \left( s^{-1} \! \left( E_{\Omega} \right) \right).
\end{equation*}
%
As with measurable spaces, the 'measurable' in measurable 
maps refers to this particular application in pushing measures 
forward from one space to another.

Furthermore, if the map is doubly-measurable then this whole 
process can be inverted: a measure $\mu_{\Omega}$ defined 
over $\left( \Omega, \EV{\Omega} \right)$ can be pushed
forward onto $\left( \Theta, \EV{\Theta} \right)$ by
%
\begin{equation*}
s^{-1}_{*} \mu_{\Omega} \! \left( E_{\Theta} \right)
\equiv
\mu_{\Omega} \! \left( s \! \left( E_{\Theta} \right) \right).
\end{equation*}
%
In other words, the same measure can be defined with respect
to any of the equivalent measurable spaces and does not 
depend on the particular details of any one of them.

\section{Probability Distributions}

As mentioned in the introduction to this chapter, we wish to quantify 
uncertainty about our descriptions of systems where logical certainty is 
impossible. In order to apply the concept of a measure, we first need to 
associate a quantity, a \emph{probability}, to each logical proposition 
whose truth or falsity may be unknown. This intuitively quantifies how 
plausible it is. To relate these to logical propositions, we assert that 
Probabilities are bounded between $0$, indicating that the statement is 
false, and $1$, indicating that the statement is true.

A \emph{probability distribution} quantifies the uncertainty about
the entire system by assigning probabilities to each event in a
given representation.  More formally, a probability distribution is
a measure assigning probabilities to each event,
%
\begin{equation*}
\pi : \mathcal{E} \! \left( \Theta \right) \rightarrow \left[0, 1 \right],
\end{equation*}
%
such that the probability of the null event is zero, 
$\pi \! \left[ \emptyset \right] = 0$, and the probability of the trivial
event is one, $\pi \! \left[ \Theta \right] = 1$.  These latter two 
conditions follow immediately from the null event always representing
a false proposition and the trivial event always representing a
true proposition.

When using a probability distribution to quantify uncertainty about 
a property taking values in the sample space $\Theta$, we often 
write $\theta \sim \pi$ which is read as, ``$\theta$ is distributed 
according to the probability distribution $\pi$'' but should really 
read ``The plausibility of any logical statement about $\theta$ 
is assigned by the probability distribution $\pi$''.

The interaction of these probability assignments and the Boolean
operations immediately give the familiar rules of probability.  For 
example, the probability of a conjunction becomes the sum rule,
%
\begin{equation*}
\pi \! \left( E_{1} \cup E_{2} \right)
= 
\pi \! \left( E_{1} \right) + \pi \! \left( E_{2} \right)
- \pi \! \left( E_{1} \cap E_{2} \right),
\end{equation*}
%
and the probability of negation becomes the rule of total probability,
%
\begin{equation*}
\pi \! \left( E \right)
= 
\pi \! \left( \Theta \right) - \pi \! \left( E^{c} \right)
=
1 - \pi \! \left( E^{c} \right).
\end{equation*}

Lebesgue integrals with respect to probability measures are better
known as \emph{expectation values},
%
\begin{equation*}
\EE_{\pi} : \mathcal{F} \! \left( \Theta \right) \rightarrow \mathbb{R},
\end{equation*}
%
Expectations include means, variances, and higher-order moments
and are the foundation of any statistical methodology.

The most important consequence of these definitions is that 
\emph{all of probability theory reduces to computing expectations}.  
Any other operation that you may have encountered in probability 
theory can only ever be an intermediate step in computing a final 
expectation.  In particular, many of the more non-intuitive aspects 
of probability theory can avoided by carefully framing everything 
as an expectation -- don't try to intuit solutions, calculate them!

\section{Pushforward Probability Distributions}

Being measures, probability distributions can also be pushed
forward from one measurable space into another using measurable
maps.

Because it is invariant when we push it forward between equivalent 
measurable spaces, a probability distribution is fundamental to the 
underlying abstract system and not any particular representation of 
that system.  In other words, different but equivalent measurable 
spaces are just different ways to describe the same system, with 
inclusion in events quantifying the same, invariant information and 
probability quantifying the same, invariant uncertainty.

\section{Conditional Probability Distributions}

\emph{Conditional probability distributions} allow us to quantify 
uncertainty about implications.  While implications assign an event 
to each value of the conditioning space, $\Phi$, a conditional 
probability distribution assigns an entire probability distribution 
to each value in the conditioning space,
%
\begin{align*}
\pi_{\Theta \mid \Phi} 
&: \EV{\Theta} \times \Phi \rightarrow \left[0, 1 \right] \\
&\quad \left( E_{\Theta}, \phi \right) \;\; \mapsto 
\pi_{\Theta \mid \Phi} \! \left( E_{\Theta} \mid \phi \right).
\end{align*}
%
In other words, for any value of $\phi \in \Phi$ the conditional 
probability distribution defines a probability distribution on $\Theta$, 
and for any event in $\Theta$ the conditional probability distribution 
defines a function from $\Phi$ to probabilities.  

\subsection{Joint Distributions and Marginal Distributions}

As with implications, conditional probability distributions can be used
to construct probability distributions on larger spaces.  In particular,
by combining a conditional probability distribution with a probability 
distribution on the conditioning space, $\Phi$, we can construct a 
probability distribution on the joint sample space, $\Theta \times \Phi$.  

This \emph{joint distribution} is defined implicitly by its probability 
assignments or expectation values.  For example, the probability of 
any joint event, $E_{\Theta} \times E_{\Phi}$, is given by first using 
the conditional probability distribution to assign a probability to 
$E_{\Theta}$ for each $\phi \in \Phi$,
%
\begin{equation*}
p \! \left( \phi \right) \equiv 
\pi_{\Theta | \Phi} \! \left( E_{\Theta} \mid \phi \right),
\end{equation*}
%
and then taking the expectation of this function with respect to the 
distribution on $\Phi$,
%
\begin{equation*}
\pi_{\Theta \times \Phi} \! \left( E_{\Theta} \times E_{\Phi} \right)
=
\mathbb{E}_{\pi_{\Phi}} \! \left[  p \cdot \mathbb{I}_{E_{\Phi}} \right],
\end{equation*}
%
where the indicator function, $\mathbb{I}_{E_{\Phi}}$, ensures that 
we take the expectation only over the event in $\Phi$.  Similarly, joint 
expectations are defined iteratively as
%
\begin{equation*}
\EE_{\pi_{\Theta \times \Phi}} \! \left[ g \! \left( \theta, \phi \right) \right]
=
\mathbb{E}_{\pi_{\Phi}} \! \left[  
\EE_{\pi_{\Theta|\Phi}} \! \Big[ 
g \! \left( \theta, \phi \right) \mid \phi 
\Big]
\right].
\end{equation*}

If we consider only the trivial event on the conditioning space $E_{\Phi} 
= \Phi$, then this construction also defines a  \emph{marginal distribution} 
on $\Theta$ by the probability assignments
%
\begin{align*}
\pi_{\Theta} \! \left( E_{\Theta} \right)
&\equiv
\pi_{\Theta \times \Phi} \! \left( E_{\Theta} \times \Phi \right) \\
&=
\mathbb{E}_{\pi_{\Phi}} \! \left[  
\pi_{\Theta|\Phi} \! \left( E_{\Theta} \mid \phi \right)
\right],
\end{align*}
or the expectations
%
\begin{align*}
\EE_{\pi_{\Theta}} \! \left[ f \! \left( \theta \right) \right]
&\equiv
\EE_{\pi_{\Theta \times \Phi}} \! \left[ f \! \left( \theta \right) \right] \\
&=
\mathbb{E}_{\pi_{\Phi}} \! \left[  
\EE_{\pi_{\Theta|\Phi}} \! \Big[ 
f \! \left( \theta \right) \mid \phi 
\Big]
\right].
\end{align*}
%
Alternatively, given the \emph{projection}
%
\begin{align*}
\varpi &: \Theta \times \Omega \rightarrow \Theta \\
& \quad \left( \theta, \omega \right) \mapsto \theta,
\end{align*}
%
we can also define this marginal distribution as the pushforward
of the joint distribution along the projection,
%
\begin{equation*}
\pi_{\Theta} = \varpi_{*} \pi_{\Theta \times \Omega}.
\end{equation*}

This marginalization process allows us to collapse a joint probability 
distribution onto any of the component spaces, explicitly incorporating 
any correlations between the components in the process.

\subsection{Bayes' Theorem}

While we can decompose any joint probability distribution into a 
conditional probability distribution and a marginal distribution, there 
is not unique decomposition.  For example, a joint probability distribution 
on $\Theta \times \Phi$ can be decomposed into $\pi_{\Theta \mid \Phi}$ 
and $\pi_{\Phi}$ or $\pi_{\Phi \mid \Theta}$ and $\pi_{\Theta}$.

The equivalence of these two decompositions immediately implies
\emph{Bayes' Theorem} which allows us to define any one of these
distributions from the other three using Radon-Nikodym derivatives. 
If $\pi_{\Theta \mid \Phi}$ and $\pi_{\Theta}$ are absolutely continuous
with respect to each other then we have
%
\begin{align*}
\pi_{\Theta \mid \Phi} &= 
\frac{ \mathrm{d} \pi_{\Phi \mid \Theta} }{ \mathrm{d} \pi_{\Phi} }
\pi_{ \Theta}
\\
\pi_{\Theta} &= 
\frac{ \mathrm{d} \pi_{\Phi} }{ \mathrm{d} \pi_{\Phi \mid \Theta} }
\pi_{ \Theta \mid \Phi }
\end{align*}
%
Similarly, if $\pi_{\Phi \mid \Theta}$ and $\pi_{\Phi}$ are absolutely 
continuous with respect to each other then we have the two additional
equalities
%
\begin{align*}
\pi_{\Phi \mid \Theta} &= 
\frac{ \mathrm{d} \pi_{\Theta \mid \Phi} }{ \mathrm{d} \pi_{\Theta} }
\pi_{\Phi}
\\
\pi_{\Theta \mid \Phi} &= 
\frac{ \mathrm{d} \pi_{\Theta} }{ \mathrm{d} \pi_{\Theta \mid \Phi} }
\pi_{ }.
\end{align*}

Although an immediate consequence of the axioms of probability
theory, Bayes' Theorem has extraordinarily important consequences 
in the application of probability theory.  These equalities allow us to 
invert any implicative structure -- if we have a probabilistic model for 
how uncertainty of propositions in $\Phi$ affects uncertainty of
propositions in $\Theta$, then Bayes'  Theorem identifies how 
uncertainty of propositions in $\Theta$ affects the uncertainty of
propositions in $\Phi$.

\subsection{Generative Modeling}

Just as we can build logical propositions sequentially from simpler
implications, we can also build probability distributions over 
high-dimensional product spaces sequentially from conditional 
probability distributions.  Mirroring the procedure for implications, 
we start with a probability distribution on one low-dimensional 
component of the target space and then build up a joint distribution 
by adding conditional probability distributions for each new component,
%
\begin{align*}
& \pi_{\Theta_{1}} \\
& \pi_{\Theta_{2} \mid \Theta_{1}} \\
& \pi_{\Theta_{3} \mid \Theta_{2}, \Theta_{1}} \\
& \cdots \\
& \pi_{\Theta_{N} \mid \Theta_{N - 1}, \ldots, \Theta_{2}, \Theta_{1}}.
\end{align*}

As with implications, these conditional probability distributions are often 
motivated by the natural structure of our target system.  In particular, if we 
think about deterministic processes as degenerate conditional probability 
distributions that assign all probability to a single event for each conditioning 
value,
%
\begin{equation*}
\pi_{\Theta \mid \Phi} \! \left[ E_{\Theta} \mid \phi \right]
= 
\left\{
\begin{array}{rr}
0, & E_{\Theta} \ne \hat{E} \! \left( \phi \right) \\
1, & E_{\Theta} = \hat{E} \! \left( \phi \right)
\end{array}
\right.,
\end{equation*}
%
then these conditional probability distributions can seamlessly incorporate
both stochastic and deterministic, causal relationships.  This iterative process 
of building a joint probability distribution from conditional probability distributions 
is the key building block of \emph{generative modeling}.
