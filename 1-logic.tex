\chapter{Logic}

Before we can formalize \emph{learning} about the world around us we 
first need a formal means of \emph{describing} the world.  \emph{Logical 
propositions} provide a mathematically-precise language for quantifying
information about a system with certainty.  In this section we will review 
abstract logical propositions, their manipulations, and how we can 
represent them in practice.  We conclude with a brief discussion of 
implicative statements.

\section{Logical Propositions}

If $\mathcal{S}$ is an abstract system of interest then any description of 
$\mathcal{S}$ can be defined as a \emph{logical proposition} which asserts
that a statement about that system is either true or false.  For example, 
consider describing one of the eight planets in our Solar System: Mercury 
($\mercury$), Venus ($\venus$), Earth ($\earth$), Mars ($\mars$), Jupiter 
($\jupiter$), Saturn ($\saturn$), Uranus ($\uranus$), and Neptune ($\neptune$).  
Information about the planet is encapsulated in logical propositions asserting
both statements that are true, for example
%
\begin{equation*}
P_{1} = \text{``the planet has an atmosphere''},
\end{equation*}
%
and statements that are false, for example
\begin{equation*} 
P_{2}=\text{``the planet does not have rings''}.
\end{equation*} 

Logical propositions can be manipulated into other logical propositions
with any of three \emph{boolean operations} (Table 
\ref{tab:boolean_truth_tables}).  \emph{Negatation}, $\neg$, is a unary 
operator resulting in a proposition asserting the converse of the input 
proposition: 
%
\begin{equation*}
\neg P_{1}=\text{``the planet does \emph{not} have an atmosphere''}.
\end{equation*}
%
Binary operations take propositions about two statements and define 
a new proposition about the joint statement.  The result of a 
\emph{conjunction}, $\wedge$, is true only when both input propositions 
are true:
%
\begin{equation*}
P_{1} \wedge P_{2} =
 \text{``the planet does not has both an atmosphere and rings''}.  
\end{equation*}
%
A \emph{disjunction}, $\vee$, however, is true whenever either input 
proposition is true : 
%
\begin{equation*}
P_{1} \vee P_{2} =
\text{``the planet does have an atmosphere \emph{or} rings''}.
\end{equation*}

\begin{table}
  \centering
  \renewcommand{\arraystretch}{1.5}
  \begin{tabular}{cc}
    \rowcolor[gray]{0.9} 
    \multicolumn{2}{c}{\textbf{Negation}} \\
    %
    \rowcolor[gray]{0.9} 
    $P$ & $\neg P$ \\
    %
    False & True \\
    True & False \\
  \end{tabular}
  %
  \vspace{3mm} \\
  %
  \begin{tabular}{ccc}
    \rowcolor[gray]{0.9} 
    \multicolumn{3}{c}{\textbf{Conjunction}} \\
    %
    \rowcolor[gray]{0.9} 
    $P_{1}$ & $P_{2}$ & $P_{1} \wedge P_{2}$ \\
    %
    False & False & False \\
    True & False & False \\
    False & True & False \\
    True & True & True \\
  \end{tabular}
  %
  \hspace{5mm}
  %
  \begin{tabular}{ccc}
    \rowcolor[gray]{0.9} 
    \multicolumn{3}{c}{\textbf{Disjunction}} \\
    %
    \rowcolor[gray]{0.9} 
    $P_{1}$ & $P_{2}$ & $P_{1} \vee P_{2}$ \\
    %
    False & False & False \\
    True & False & True \\
    False & True & True \\
    True & True & True \\
  \end{tabular}
\caption{The three basic Boolean operations transform given
logical propositions into new logical propositions, and their 
action is conveniently summarized in truth tables.}
\label{tab:boolean_truth_tables}
\end{table}

Naively, we might think that a system would be completely described only 
once we have defined a proposition for every statement, classifying each 
as either true or false. In his infamous Incompleteness Theorems, however, 
G\"{o}del showed that if we try to define a proposition for every possible 
statement about $\mathcal{S}$ then there will exist some pathological 
propositions that are not consistent with the Boolean operations.  For 
example, there are always pathological propositions asserting truth whose 
negations do not assert falseness!  These pathological propositions cannot 
be explicitly constructed in practice, only proven to exist, but in order to 
ensure a self-consistent description of the system we need some means of 
removing them.

To avoid these pathological propositions we leverage the fact that propositions
about non-pathological statements yield non-pathological propositions when 
we apply any Boolean operation.  In other words, well-behaved propositions 
are \emph{closed} under these operations, and any collection of well-behaved, 
consistent propositions forms a logical algebra, 
$\mathcal{L} \! \left( \mathcal{S} \right)$.  

We will refer to the pairing of an abstract system with some choice of a logical
algebra as a \emph{logical model}, 
$\left( \mathcal{S}, \mathcal{L} \! \left( \mathcal{S} \right) \right)$.

\section{Representations of Logical Models}

Logical models provide a very abstract, and hence very genetic, framework for 
describing arbitrary systems, but that abstraction also makes these descriptions
ungainly to specify and manipulate in practice.  For our abstract descriptions to 
be useful we need an explicit language in which we can communicate them, 
which we will refer to as a \emph{representation} 
(Table \ref{tab:representation_examples}).

\begin{table}
  \centering
  \renewcommand{\arraystretch}{1.5}
  \begin{tabular}{ccc}
    \rowcolor[gray]{0.9} 
    \textbf{Abstract System} & \textbf{Representation} & \textbf{Measurable Maps} \\
    %
    Historical Decrees & Ancient Languages & Translations \\
    & (Egyptian, Demotic, Green, $\ldots$) & (Rosetta Stone) \\
    % 
    Computer Program & Programming Languages & Source-to-Source \\
    & (C++, S, Python, $\ldots$) & Compilers \\
    %
    Unordered Categories & Integers & Permutations \\
    %
    Direction & Vector Space & Changes of Basis \\
    %
    Position & Coordinates & Reparameterizations \\
    %
    Quantification & Units & Changes of Units \\
  \end{tabular}
\caption{Representations provide a concrete language in which we can
communicate the descriptions of an abstract system.  Because there 
are many degenerate ways to express these descriptions, these 
representations are not unique and in practice we have to be careful 
that the irrelevant details of any particular representation do not affect
the descriptions themselves.}
\label{tab:representation_examples}
\end{table}

In order to construct a representation we assume that the properties
of our abstract system take values in some \emph{sample space}, 
$\Theta$, such as integers or real numbers.  Statements about these
properties are represented as subsets of the sample space, and
logical propositions are represented as set inclusion.  If a statement
is represented by $E \subset \Theta$, for example, then a logical
proposition asserting that the statement is true is represented by 
asserting that $\theta \in E$, and one asserting that the statement
is false is represented by asserting that $\theta \notin E$.

With propositions represented by set inclusion and set exclusion,
the Boolean operations are naturally implemented with set operations.  
Conjunction becomes inclusion in the subset intersection, disjunction 
inclusion in the subset union, and negation inclusion in the subset 
complement negation (Figure \ref{fig:venn_diagrams}, 
Table \ref{tab:set_truth_tables}).  

\begin{figure*}
\centering
\subfigure[]{
\begin{tikzpicture}[scale=0.3, thick]
  \draw[color=white] (-0.5, 0) to (20.5, 0);
  \draw[color=black] (0, 0) to (20, 0) to (20, 15) to (0, 15) to (0, 0);
  \draw[color=gray80, dashed] (12, 9) circle (150pt); 
  \fill[color=dark] (8, 6) circle (150pt) node[color=white] {$E_{1}$}; 
\end{tikzpicture}
%
\begin{tikzpicture}[scale=0.3, thick]
  \draw[color=white] (-0.5, 0) to (20.5, 0);
  \draw[color=black] (0, 0) to (20, 0) to (20, 15) to (0, 15) to (0, 0);
  \draw[color=gray80, dashed] (8, 6) circle (150pt); 
  \fill[color=dark] (12, 9) circle (150pt) node[color=white] {$E_{2}$}; 
\end{tikzpicture}
}
%
\subfigure[]{
\begin{tikzpicture}[scale=0.3, thick]
  \draw[color=black] (0, 0) to (20, 0) to (20, 15) to (0, 15) to (0, 0);
  \draw[color=gray80, dashed] (8, 6) circle (150pt); 
  \draw[color=gray80, dashed] (12, 9) circle (150pt); 
  \begin{scope}
    \clip (8, 6) circle (150pt);
    \fill[color=dark] (12, 9) circle (150pt); 
  \end{scope}
  \node[color=white] at (10, 7.5) {$E_{1} \cap E_{2}$};
\end{tikzpicture}
}
%
\subfigure[]{
\begin{tikzpicture}[scale=0.3, thick]
  \draw[color=black] (0, 0) to (20, 0) to (20, 15) to (0, 15) to (0, 0);
  \fill[color=dark] (8, 6) circle (150pt); 
  \fill[color=dark] (12, 9) circle (150pt); 
  \node[color=white] at (10, 7.5) {$E_{1} \cup E_{2}$};
\end{tikzpicture}
}
%
\subfigure[]{
\begin{tikzpicture}[scale=0.3, thick]
  \fill[color=dark] (0, 0) to (20, 0) to (20, 15) to (0, 15) to (0, 0);
  \draw[color=black] (0, 0) to (20, 0) to (20, 15) to (0, 15) to (0, 0);
  \fill[color=white] (8, 6) circle (150pt); 
  \node[color=white] at (16, 11) {$E_{1}^{c}$};
\end{tikzpicture}
}
%
\caption{In a representation statements are defined as subsets in some
sample space and logical propositions are implemented with set inclusion
or set exclusion.  Boolean operations are then implemented with inclusion
in a set complement (negation), set intersection (conjunction), and set
union (disjunction).  (a) If two propositions are represented by
inclusion in the sets $E_{1}$ and $E_{2}$, then (b) their conjunction is 
represented as inclusion in the set intersection, $E_{1} \cup E_{2}$, and (c) 
their disjunction is represented as inclusion in the set union, 
$E_{1} \cap E_{2}$.  (d) Similarly, the negation of $E_{1}$ is represented
as inclusion in the set complement, $E_{1}^{c}$.}
\label{fig:venn_diagrams}
\end{figure*}

\begin{table}
  \centering
  \renewcommand{\arraystretch}{1.5}
  \begin{tabular}{cc}
    \rowcolor[gray]{0.9} 
    \multicolumn{2}{c}{\textbf{Negation}} \\
    %
    \rowcolor[gray]{0.9} 
    $\theta \in E$ & $\theta \in E^{c}$ \\
    %
    False & True \\
    True & False \\
  \end{tabular}
  %
  \vspace{3mm} \\
  %
  \begin{tabular}{ccc}
    \rowcolor[gray]{0.9} 
    \multicolumn{3}{c}{\textbf{Conjunction}} \\
    %
    \rowcolor[gray]{0.9} 
    $\theta \in E_{1}$ & $\theta \in E_{1}$ & $\theta \in E_{1} \cap E_{2}$ \\
    %
    False & False & False \\
    True & False & False \\
    False & True & False \\
    True & True & True \\
  \end{tabular}
  %
  \hspace{5mm}
  %
  \begin{tabular}{ccc}
    \rowcolor[gray]{0.9} 
    \multicolumn{3}{c}{\textbf{Disjunction}} \\
    %
    \rowcolor[gray]{0.9} 
    $\theta \in E_{1}$ & $\theta \in E_{1}$ & $\theta \in E_{1} \cup E_{2}$ \\
    %
    False & False & False \\
    True & False & True \\
    False & True & True \\
    True & True & True \\
  \end{tabular}
\caption{In a representation statements are defined as subsets in some
sample space and logical propositions are implemented with set inclusion
or set exclusion.  Boolean operations are then implemented with inclusion
in a set complement (negation), set intersection (conjunction), and set
union (disjunction), which yield the same truth tables and hence define
a valid representation.}
\label{tab:set_truth_tables}
\end{table}

The same inconsistencies that limited us to consider only well-behaved 
logical propositions, however, also limit our representations to only
well-behaved subsets of the sample space.  We denote a collection of 
subsets of the sample space that is closed under set compliments,
set intersections, and set unions as an \emph{event space}, $\EV{\Theta}$,
with the well-behaved subsets in that space denoted \emph{events}. 
Event spaces always include the null event, $E = \emptyset$, which 
can never be true as it is always empty.  Similarly, event spaces also 
always include the trivial event, $E = \Theta$, which can never be false
as every element of the sample space is included in this event.  
The pairing of a sample space and an event space is denoted a
\emph{measurable space}, for reasons that will become more clear
in the next chapter.

More formally, then, we should define a representation as a map from 
a logical model into a measurable space.  In practice this is defined 
as a map from the abstract system into the sample space,
%
\begin{equation*}
\mathfrak{r} : \mathcal{S} \rightarrow \Theta,
\end{equation*}
%
that then induces a map from the logical algebra into the event space,
%
\begin{equation*}
\mathfrak{r}_{*} : 
\mathcal{L} \! \left( \mathcal{S} \right) \rightarrow \EV{\Theta}.
\end{equation*}
%
A representation is called \emph{faithful} if every logical statement maps 
into an event and vice-versa, and Stone's Theorem guarantees that a
faithful representation for any logical model always exists.

For example, descriptions of the planets in our solar system can be 
represented with the integers, $\Theta = \left\{1, 2, \ldots, 8 \right\}$, 
with each logical proposition represented by inclusion in a subset of 
those integers (Figure \ref{fig:discrete_set_logic}).  Similarly, the 
distance between two objects can be represented by real numbers,
(Figure \ref{fig:real_set_logic}).

\begin{figure*}
\centering
\subfigure[]{
\begin{tikzpicture}[scale=0.3, thick]
  \draw[color=white] (-25, 0) to (10, 0);

  \node[] at (-15, 0) {The planet has a magnetic field};
  \node[] at (7, 2) {$\theta \in E_{1}$};

  \fill[color=gray80] (0, 0) circle (25pt) node[color=white] {$\mercury$}; 
  \fill[color=dark] (2, 0) circle (25pt) node[color=white] {$\venus$}; 
  \fill[color=dark] (4, 0) circle (25pt) node[color=white] {$\earth$}; 
  \fill[color=gray80] (6, 0) circle (25pt) node[color=white] {$\mars$}; 
  \fill[color=dark] (8, 0) circle (25pt) node[color=white] {$\jupiter$}; 
  \fill[color=dark] (10, 0) circle (25pt) node[color=white] {$\saturn$}; 
  \fill[color=dark] (12, 0) circle (25pt) node[color=white] {$\uranus$}; 
  \fill[color=dark] (14, 0) circle (25pt) node[color=white] {$\neptune$};  
  
  \node[] at (-15, -5) {The planet has moons};
  \node[] at (7, -3) {$\theta \in E_{2}$};

  \fill[color=gray80] (0, -5) circle (25pt) node[color=white] {$\mercury$}; 
  \fill[color=gray80] (2, -5) circle (25pt) node[color=white] {$\venus$}; 
  \fill[color=dark] (4, -5) circle (25pt) node[color=white] {$\earth$}; 
  \fill[color=dark] (6, -5) circle (25pt) node[color=white] {$\mars$}; 
  \fill[color=dark] (8, -5) circle (25pt) node[color=white] {$\jupiter$}; 
  \fill[color=dark] (10, -5) circle (25pt) node[color=white] {$\saturn$}; 
  \fill[color=dark] (12, -5) circle (25pt) node[color=white] {$\uranus$}; 
  \fill[color=dark] (14, -5) circle (25pt) node[color=white] {$\neptune$};  
\end{tikzpicture}
}
%
\subfigure[]{
\begin{tikzpicture}[scale=0.3, thick]
  \draw[color=white] (-25, 0) to (10, 0);
  
  \node[align=center] at (-15, 0) {The planet has a magnetic field\\and moons};
  \node[] at (7, 2) {$\theta \in E_{1} \cap E_{2}$};

  \fill[color=gray80] (0, 0) circle (25pt) node[color=white] {$\mercury$}; 
  \fill[color=gray80] (2, 0) circle (25pt) node[color=white] {$\venus$}; 
  \fill[color=dark] (4, 0) circle (25pt) node[color=white] {$\earth$}; 
  \fill[color=gray80] (6, 0) circle (25pt) node[color=white] {$\mars$}; 
  \fill[color=dark] (8, 0) circle (25pt) node[color=white] {$\jupiter$}; 
  \fill[color=dark] (10, 0) circle (25pt) node[color=white] {$\saturn$}; 
  \fill[color=dark] (12, 0) circle (25pt) node[color=white] {$\uranus$}; 
  \fill[color=dark] (14, 0) circle (25pt) node[color=white] {$\neptune$};  
\end{tikzpicture}
}
%
\subfigure[]{
\begin{tikzpicture}[scale=0.3, thick]
  \draw[color=white] (-25, 0) to (10, 0);
   
  \node[align=center] at (-15, 0) {The planet has a magnetic field\\or moons};
  \node[] at (7, 2) {$\theta \in E_{1} \cup E_{2}$};

  \fill[color=gray80] (0, 0) circle (25pt) node[color=white] {$\mercury$}; 
  \fill[color=dark] (2, 0) circle (25pt) node[color=white] {$\venus$}; 
  \fill[color=dark] (4, 0) circle (25pt) node[color=white] {$\earth$}; 
  \fill[color=dark] (6, 0) circle (25pt) node[color=white] {$\mars$}; 
  \fill[color=dark] (8, 0) circle (25pt) node[color=white] {$\jupiter$}; 
  \fill[color=dark] (10, 0) circle (25pt) node[color=white] {$\saturn$}; 
  \fill[color=dark] (12, 0) circle (25pt) node[color=white] {$\uranus$}; 
  \fill[color=dark] (14, 0) circle (25pt) node[color=white] {$\neptune$};  
\end{tikzpicture}
}
%
\subfigure[]{
\begin{tikzpicture}[scale=0.3, thick]
  \draw[color=white] (-25, 0) to (10, 0);
   
  \node[align=center] at (-15, 0) {The planet does not\\have a magnetic field};
  \node[] at (7, 2) {$\theta \in E_{1}^{c}$};

  \fill[color=dark] (0, 0) circle (25pt) node[color=white] {$\mercury$}; 
  \fill[color=gray80] (2, 0) circle (25pt) node[color=white] {$\venus$}; 
  \fill[color=gray80] (4, 0) circle (25pt) node[color=white] {$\earth$}; 
  \fill[color=dark] (6, 0) circle (25pt) node[color=white] {$\mars$}; 
  \fill[color=gray80] (8, 0) circle (25pt) node[color=white] {$\jupiter$}; 
  \fill[color=gray80] (10, 0) circle (25pt) node[color=white] {$\saturn$}; 
  \fill[color=gray80] (12, 0) circle (25pt) node[color=white] {$\uranus$}; 
  \fill[color=gray80] (14, 0) circle (25pt) node[color=white] {$\neptune$};  
\end{tikzpicture}
}
\caption{(a) Systems that take discrete values, such as the planets in 
our solar system, can be represented with integers, with any logical 
proposition about the planets represented by inclusion in a subset of 
integers. (b) Conjuction of logical propositions is implemented with 
inclusion in set intersections, (c) disjunction with inclusion in set unions, 
and (d) negation with inclusion in set complements.}
\label{fig:discrete_set_logic}
\end{figure*}

\begin{figure*}
\centering
\subfigure[]{
\begin{tikzpicture}[scale=0.3, thick]
  \draw[color=white] (-27, 0) to (17, 0);

  \node[align=center] at (-15, 0) {The distance is\\less than five inches};
  \node[] at (7.5, 2) {$\theta \in E_{1}$};

  \draw[|->] (0, 0) -- (14,0) node[right] {$x$};
  \draw[line width=1mm, color=dark] (0, 0) node[] {$\,($} -- (5, 0) node[] {$\!)$};
  
  \node[align=center] at (-15, -5) {The distance is between\\three and seven inches};
  \node[] at (7.5, -3) {$\theta \in E_{2}$};
  
  \draw[|->] (0, -5) -- (14,-5) node[right] {$x$};
  \draw[line width=1mm, color=dark] (3, -5) node[] {$\,($} -- (7,-5) node[] {$\!)$};

\end{tikzpicture}
}
%
\subfigure[]{
\begin{tikzpicture}[scale=0.3, thick]
  \draw[color=white] (-27, 0) to (17, 0);
  
  \node[align=center] at (-15, 0) {The distance is less than five inches\\and
                                                   between three and seven inches};
  \node[] at (7.5, 2) {$\theta \in E_{1} \cap E_{2}$};

  \draw[|->] (0, 0) -- (14,0) node[right] {$x$};
  \draw[line width=1mm, color=dark] (3, 0) node[] {$\,($} -- (5, 0) node[] {$\!)$};
\end{tikzpicture}
}
%
\subfigure[]{
\begin{tikzpicture}[scale=0.3, thick]
  \draw[color=white] (-27, 0) to (17, 0);
   
  \node[align=center] at (-15, 0) {The distance is less than five inches\\or
                                                   between three and seven inches};
  \node[] at (7.5, 2) {$\theta \in E_{1} \cup E_{2}$};

  \draw[|->] (0, 0) -- (14, 0) node[right] {$x$};
  \draw[line width=1mm, color=dark] (0, 0) node[] {$\,($} -- (7, 0) node[] {$\!)$};
\end{tikzpicture}
}
%
\subfigure[]{
\begin{tikzpicture}[scale=0.3, thick]
  \draw[color=white] (-27, 0) to (17, 0);
   
  \node[align=center] at (-15, 0) {The distance is not\\less than five inches};
  \node[] at (7.5, 2) {$\theta \in E_{1}^{c}$};

  \draw[|->] (0, 0) -- (14, 0) node[right] {$x$};
  \draw[line width=1mm, color=dark] (5, 0) node[] {$\,($} -- (13, 0);
\end{tikzpicture}
}
\caption{(a) Systems that take continuous values, such as distances, 
can be represented with real numbers, with any logical proposition
represented by inclusion in well-behaved subsets of real numbers. 
(b) Conjuction of logical propositions is implemented with inclusion 
in set intersections, (c) disjunction with inclusion in set unions, and 
(d) negation with inclusion in set complements.}
\label{fig:real_set_logic}
\end{figure*}

\section{Equivalent Representations}

Because they map abstract systems into explicit contexts, representations
are critical to the practical implementation of logical models and their
manipulations.  Unfortunately representations also bring with them a 
dangerous subtlety -- there is no \emph{unique} representation of a 
given abstract system.

A map from one sample space into another, $s : \Theta \rightarrow \Omega$,
is \emph{measurable} if every event in $\Omega$ is generated by an event 
in $\Theta$,
%
\begin{equation*}
s^{-1} \! \left( E_{\Omega} \right) \in \EV{\Theta}.
\end{equation*}
%
Measurable maps allow us to completely generate the measurable space 
$\left( \Omega, \EV{\Omega} \right)$ by mapping forward elements of the 
initial measurable space $\left(\Theta, \EV{\Theta} \right)$; we cannot, 
however, necessarily achieve the converse as there is no guarantee that 
every event in $\EV{\Theta}$ maps to an event in $\EV{\Omega}$.
This process is also known as \emph{pushing forward} the measurable 
space $\left( \Theta, \EV{\Theta} \right)$ into the measurable space 
$\left( \Omega, \EV{\Omega} \right)$, and in this context  
$\left( \Omega, \EV{\Omega} \right)$ is denoted the \emph{pushforward} 
of $\left( \Theta, \EV{\Theta} \right)$ by $s$.

Composing a faithful representation with a measurable map,
%
\begin{equation*}
s \circ \mathfrak{e} : \mathcal{S} \rightarrow \Theta \rightarrow \Omega,
\end{equation*}
%
yields another valid representation but not necessarily a faithful one.
Because there is no guarantee that every event in $\EV{\Theta}$ maps
to an event in $\EV{\Omega}$, this pushforward may be able to represent
only a handful of propositions in the logical model.  For example,
a measurable map might round real numbers down to discrete numbers,
losing the ability to represent all but a countable number of logical 
propositions.

In order to maintain a faithful representation we need to ensure that
the events spaces are preserved by both the map and its inverse.  Here
we will say that a map $s : \Theta \rightarrow \Omega$ is 
\emph{doubly-measurable} if every event in $\EV{\Theta}$ maps to an 
event in $\EV{\Omega}$ and vice versa,
%
\begin{align*}
s \! \left( E_{\Theta} \right) &\in \EV{\Omega}
\\
s^{-1} \! \left( E_{\Omega} \right) &\in \EV{\Theta}.
\end{align*}
%
A doubly-measurable map allows us to recover either measurable space 
as a pushforward of the other, and the composition of a faithful 
representation with a doubly-measurable map,
%
\begin{equation*}
s \circ \mathfrak{e} : \mathcal{S} \rightarrow \Theta \rightarrow \Omega,
\end{equation*}
%
is then another faithful representation.  We warn the reader that there
is no conventional terminology for these maps and we have introduced
doubly-measurable here as our own terminology.

When a doubly-measurable map exists between two measurable spaces
then the two spaces can be used to specify the same logical model,
and we say that they are \emph{equivalent representations} as they provide
equivalent descriptions of a given system 
(Table \ref{tab:representation_examples}).  The set of measurable spaces
equivalent to $\left( \Theta, \EV{\Theta} \right)$ will be denoted
$\lfloor \Theta, \EV{\Theta} \rfloor$.

For example, if the sample space is discrete then we can always define 
an equivalent measurable space by simply permuting the labels.  It doesn't 
matter if we order the planets by distance from the sun, by diameter, or 
by any other metric: the measurable spaces quantify the same information 
(Figure \ref{fig:permuting_discrete_spaces}).   Similarly, we can always 
apply a transformation that warps real numbers to map between real 
sample spaces without compromising our ability to represent propositions 
with inclusion in events.  One of the most common ways this manifests in 
practice is when our descriptions require units.  The information we quantify 
doesn't depend on whether we express distance in {\aa}ngstr\"{o}ms or 
inches or meters or furlongs: each unit defines a separate but equivalent 
sample space.

\begin{figure*}
\centering
\begin{tikzpicture}[scale=0.3, thick]
  \foreach \i in {1, 2, ..., 8} {
    \node[] at ({2 * \i - 2}, 2) {$\i$};
  }  

  \fill[color=dark] (0, 0) circle (25pt) node[color=white] {$\mercury$}; 
  \fill[color=dark] (2, 0) circle (25pt) node[color=white] {$\venus$}; 
  \fill[color=dark] (4, 0) circle (25pt) node[color=white] {$\earth$}; 
  \fill[color=dark] (6, 0) circle (25pt) node[color=white] {$\mars$}; 
  \fill[color=dark] (8, 0) circle (25pt) node[color=white] {$\jupiter$}; 
  \fill[color=dark] (10, 0) circle (25pt) node[color=white] {$\saturn$}; 
  \fill[color=dark] (12, 0) circle (25pt) node[color=white] {$\uranus$}; 
  \fill[color=dark] (14, 0) circle (25pt) node[color=white] {$\neptune$};  
  
  \draw[->] (0, -1) -- (0, -5);
  \draw[->] (2, -1) -- (4, -5);
  \draw[->] (4, -1) -- (6, -5);
  \draw[->] (6, -1) -- (2, -5);
  \draw[->] (8, -1) -- (14, -5);
  \draw[->] (10, -1) -- (12, -5);
  \draw[->] (12, -1) -- (10, -5);
  \draw[->] (14, -1) -- (8, -5);
  
  \fill[color=dark] (0, -6) circle (25pt) node[color=white] {$\mercury$}; 
  \fill[color=dark] (2, -6) circle (25pt) node[color=white] {$\mars$}; 
  \fill[color=dark] (4, -6) circle (25pt) node[color=white] {$\venus$}; 
  \fill[color=dark] (6, -6) circle (25pt) node[color=white] {$\earth$}; 
  \fill[color=dark] (8, -6) circle (25pt) node[color=white] {$\neptune$};  
  \fill[color=dark] (10, -6) circle (25pt) node[color=white] {$\uranus$};
  \fill[color=dark] (12, -6) circle (25pt) node[color=white] {$\saturn$};  
  \fill[color=dark] (14, -6) circle (25pt) node[color=white] {$\jupiter$}; 

  \foreach \i in {1, 2, ..., 8} {
    \node[] at ({2 * \i - 2}, -8) {$\i$};
  }  
          
\end{tikzpicture}
\caption{Discrete sample spaces are not unique way to express a system 
as we can always permute the arbitrary labels to yield an equivalent sample 
space.  For example, we could order the planets by distance from the sun 
or diameter without affecting our ability to describe the planets themselves.}
\label{fig:permuting_discrete_spaces}
\end{figure*}

Ultimately, there are many ways to describe of a given logical model and 
measurable maps translate from one equivalent representation to another.  
Although certain representations may be more useful than others in practice, 
we have to be careful to ensure that our final analysis does not depend on 
the irrelevant details of any particular representation.

\section{Implications}

There are various Boolean operations that we can derive from the basic 
operations of negation, conjunction, and disjunction, and one of the most
powerful are \emph{implications}, which assert the truth of a statement
conditioned on the truth of another statement.  For example, we might not 
be able to assert that ``a planet has a magnetic field'' universally, but we 
might be able to assert the implication ``if a planet has a rotating core then 
it has a magnetic field''.

Implications are particularly naturally when we want to describe a target
system conditioned on the state of some other system.  Letting 
$\left(\Theta, \EV{\Theta} \right)$ be a measurable space capable of 
representing the target system and $\left(\Phi, \EV{\Phi} \right)$ be a 
measurable space capable of representing the conditioning system, an 
implication is represented as a map from points in $\Phi$ to events in 
$\EV{\Theta}$,
%
\begin{align*}
\iota_{\Theta \mid \Phi} :& \, \Phi \rightarrow \EV{\Theta}
\\
& \, \phi \mapsto \iota_{\Theta \mid \Phi} \! \left( \phi \right).
\end{align*}
%
For any event in the conditioning space, $E_{\Phi} \in \EV{\Phi}$,
the implication also defines an event in the joint sample space, 
$\Theta \times \Phi$, as the union of all of the events implied by each 
point in $E_{\Phi}$,
%
\begin{align*}
\iota^{*}_{\Theta \mid \Phi} 
&: \EV{\Phi} \rightarrow \EV{\Theta \times \Phi}
\\
& \quad \; E_{\Phi} \; \mapsto
\bigcup_{\phi \in E_{\Phi} } \iota_{\Theta \mid \Phi} \! \left( \phi \right).
\end{align*}

Because relationships are often much easier to reason about than the 
entirety of a complex system, these induced events can drastically facilitate
the description of complex systems by decomposing logical propositions 
into simpler implications.  For example, instead of having to enumerate all 
possible combinations of planetary core behavior and magnetic field, we 
can utilize the single implication that only a rotating core generates a 
magnetic field to build any logical proposition about the joint core-magnetic 
field system from a logical proposition about the core alone.  

More formally, any logical proposition representation in a joint sample space 
$\prod_{n = 1}^{N} \Theta_{n}$ can be built up sequentially, starting with an 
event in one component and adding implications one after another,
%
\begin{align*}
& E_{\Theta_{1}} \\
& \iota^{*}_{\Theta_{2} \mid \Theta_{1}} \! \left( E_{\Theta_{1}} \right) \\
& \iota^{*}_{\Theta_{3} \mid \Theta_{2}, \Theta_{1}} 
\circ
\iota^{*}_{\Theta_{2} \mid \Theta_{1}} \! \left( E_{\Theta_{1}} \right) \\
& \cdots \\
& \iota^{*}_{\Theta_{N} \mid \Theta_{N - 1}, \ldots, \Theta_{2}, \Theta_{1}}
\circ
\iota^{*}_{\Theta_{3} \mid \Theta_{2}, \Theta_{1}}  
\circ
\iota^{*}_{\Theta_{2} \mid \Theta_{1}} \! \left( E_{\Theta_{1}} \right).
\end{align*}
